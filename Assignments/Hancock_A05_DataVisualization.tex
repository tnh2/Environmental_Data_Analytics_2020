% Options for packages loaded elsewhere
\PassOptionsToPackage{unicode}{hyperref}
\PassOptionsToPackage{hyphens}{url}
%
\documentclass[
]{article}
\usepackage{lmodern}
\usepackage{amssymb,amsmath}
\usepackage{ifxetex,ifluatex}
\ifnum 0\ifxetex 1\fi\ifluatex 1\fi=0 % if pdftex
  \usepackage[T1]{fontenc}
  \usepackage[utf8]{inputenc}
  \usepackage{textcomp} % provide euro and other symbols
\else % if luatex or xetex
  \usepackage{unicode-math}
  \defaultfontfeatures{Scale=MatchLowercase}
  \defaultfontfeatures[\rmfamily]{Ligatures=TeX,Scale=1}
\fi
% Use upquote if available, for straight quotes in verbatim environments
\IfFileExists{upquote.sty}{\usepackage{upquote}}{}
\IfFileExists{microtype.sty}{% use microtype if available
  \usepackage[]{microtype}
  \UseMicrotypeSet[protrusion]{basicmath} % disable protrusion for tt fonts
}{}
\makeatletter
\@ifundefined{KOMAClassName}{% if non-KOMA class
  \IfFileExists{parskip.sty}{%
    \usepackage{parskip}
  }{% else
    \setlength{\parindent}{0pt}
    \setlength{\parskip}{6pt plus 2pt minus 1pt}}
}{% if KOMA class
  \KOMAoptions{parskip=half}}
\makeatother
\usepackage{xcolor}
\IfFileExists{xurl.sty}{\usepackage{xurl}}{} % add URL line breaks if available
\IfFileExists{bookmark.sty}{\usepackage{bookmark}}{\usepackage{hyperref}}
\hypersetup{
  pdftitle={Assignment 5: Data Visualization},
  pdfauthor={Thomas Hancock},
  hidelinks,
  pdfcreator={LaTeX via pandoc}}
\urlstyle{same} % disable monospaced font for URLs
\usepackage[margin=2.54cm]{geometry}
\usepackage{color}
\usepackage{fancyvrb}
\newcommand{\VerbBar}{|}
\newcommand{\VERB}{\Verb[commandchars=\\\{\}]}
\DefineVerbatimEnvironment{Highlighting}{Verbatim}{commandchars=\\\{\}}
% Add ',fontsize=\small' for more characters per line
\usepackage{framed}
\definecolor{shadecolor}{RGB}{248,248,248}
\newenvironment{Shaded}{\begin{snugshade}}{\end{snugshade}}
\newcommand{\AlertTok}[1]{\textcolor[rgb]{0.94,0.16,0.16}{#1}}
\newcommand{\AnnotationTok}[1]{\textcolor[rgb]{0.56,0.35,0.01}{\textbf{\textit{#1}}}}
\newcommand{\AttributeTok}[1]{\textcolor[rgb]{0.77,0.63,0.00}{#1}}
\newcommand{\BaseNTok}[1]{\textcolor[rgb]{0.00,0.00,0.81}{#1}}
\newcommand{\BuiltInTok}[1]{#1}
\newcommand{\CharTok}[1]{\textcolor[rgb]{0.31,0.60,0.02}{#1}}
\newcommand{\CommentTok}[1]{\textcolor[rgb]{0.56,0.35,0.01}{\textit{#1}}}
\newcommand{\CommentVarTok}[1]{\textcolor[rgb]{0.56,0.35,0.01}{\textbf{\textit{#1}}}}
\newcommand{\ConstantTok}[1]{\textcolor[rgb]{0.00,0.00,0.00}{#1}}
\newcommand{\ControlFlowTok}[1]{\textcolor[rgb]{0.13,0.29,0.53}{\textbf{#1}}}
\newcommand{\DataTypeTok}[1]{\textcolor[rgb]{0.13,0.29,0.53}{#1}}
\newcommand{\DecValTok}[1]{\textcolor[rgb]{0.00,0.00,0.81}{#1}}
\newcommand{\DocumentationTok}[1]{\textcolor[rgb]{0.56,0.35,0.01}{\textbf{\textit{#1}}}}
\newcommand{\ErrorTok}[1]{\textcolor[rgb]{0.64,0.00,0.00}{\textbf{#1}}}
\newcommand{\ExtensionTok}[1]{#1}
\newcommand{\FloatTok}[1]{\textcolor[rgb]{0.00,0.00,0.81}{#1}}
\newcommand{\FunctionTok}[1]{\textcolor[rgb]{0.00,0.00,0.00}{#1}}
\newcommand{\ImportTok}[1]{#1}
\newcommand{\InformationTok}[1]{\textcolor[rgb]{0.56,0.35,0.01}{\textbf{\textit{#1}}}}
\newcommand{\KeywordTok}[1]{\textcolor[rgb]{0.13,0.29,0.53}{\textbf{#1}}}
\newcommand{\NormalTok}[1]{#1}
\newcommand{\OperatorTok}[1]{\textcolor[rgb]{0.81,0.36,0.00}{\textbf{#1}}}
\newcommand{\OtherTok}[1]{\textcolor[rgb]{0.56,0.35,0.01}{#1}}
\newcommand{\PreprocessorTok}[1]{\textcolor[rgb]{0.56,0.35,0.01}{\textit{#1}}}
\newcommand{\RegionMarkerTok}[1]{#1}
\newcommand{\SpecialCharTok}[1]{\textcolor[rgb]{0.00,0.00,0.00}{#1}}
\newcommand{\SpecialStringTok}[1]{\textcolor[rgb]{0.31,0.60,0.02}{#1}}
\newcommand{\StringTok}[1]{\textcolor[rgb]{0.31,0.60,0.02}{#1}}
\newcommand{\VariableTok}[1]{\textcolor[rgb]{0.00,0.00,0.00}{#1}}
\newcommand{\VerbatimStringTok}[1]{\textcolor[rgb]{0.31,0.60,0.02}{#1}}
\newcommand{\WarningTok}[1]{\textcolor[rgb]{0.56,0.35,0.01}{\textbf{\textit{#1}}}}
\usepackage{graphicx,grffile}
\makeatletter
\def\maxwidth{\ifdim\Gin@nat@width>\linewidth\linewidth\else\Gin@nat@width\fi}
\def\maxheight{\ifdim\Gin@nat@height>\textheight\textheight\else\Gin@nat@height\fi}
\makeatother
% Scale images if necessary, so that they will not overflow the page
% margins by default, and it is still possible to overwrite the defaults
% using explicit options in \includegraphics[width, height, ...]{}
\setkeys{Gin}{width=\maxwidth,height=\maxheight,keepaspectratio}
% Set default figure placement to htbp
\makeatletter
\def\fps@figure{htbp}
\makeatother
\setlength{\emergencystretch}{3em} % prevent overfull lines
\providecommand{\tightlist}{%
  \setlength{\itemsep}{0pt}\setlength{\parskip}{0pt}}
\setcounter{secnumdepth}{-\maxdimen} % remove section numbering

\title{Assignment 5: Data Visualization}
\author{Thomas Hancock}
\date{}

\begin{document}
\maketitle

\hypertarget{overview}{%
\subsection{OVERVIEW}\label{overview}}

This exercise accompanies the lessons in Environmental Data Analytics on
Data Visualization

\hypertarget{directions}{%
\subsection{Directions}\label{directions}}

\begin{enumerate}
\def\labelenumi{\arabic{enumi}.}
\tightlist
\item
  Change ``Student Name'' on line 3 (above) with your name.
\item
  Work through the steps, \textbf{creating code and output} that fulfill
  each instruction.
\item
  Be sure to \textbf{answer the questions} in this assignment document.
\item
  When you have completed the assignment, \textbf{Knit} the text and
  code into a single PDF file.
\item
  After Knitting, submit the completed exercise (PDF file) to the
  dropbox in Sakai. Add your last name into the file name (e.g.,
  ``Salk\_A05\_DataVisualization.Rmd'') prior to submission.
\end{enumerate}

The completed exercise is due on Tuesday, February 11 at 1:00 pm.

\hypertarget{set-up-your-session}{%
\subsection{Set up your session}\label{set-up-your-session}}

\begin{enumerate}
\def\labelenumi{\arabic{enumi}.}
\item
  Set up your session. Verify your working directory and load the
  tidyverse and cowplot packages. Upload the NTL-LTER processed data
  files for nutrients and chemistry/physics for Peter and Paul Lakes
  (tidy and gathered) and the processed data file for the Niwot Ridge
  litter dataset.
\item
  Make sure R is reading dates as date format; if not change the format
  to date.
\end{enumerate}

\begin{Shaded}
\begin{Highlighting}[]
\CommentTok{#1}
\KeywordTok{getwd}\NormalTok{()}
\end{Highlighting}
\end{Shaded}

\begin{verbatim}
## [1] "C:/Users/thoma/Thomas/2018 Grad School/Duke MEM/ENV 872/Environmental_Data_Analytics_2020"
\end{verbatim}

\begin{Shaded}
\begin{Highlighting}[]
\KeywordTok{library}\NormalTok{(tidyverse)}
\end{Highlighting}
\end{Shaded}

\begin{verbatim}
## -- Attaching packages --------------------------------------------------------- tidyverse 1.3.0 --
\end{verbatim}

\begin{verbatim}
## v ggplot2 3.2.1     v purrr   0.3.3
## v tibble  2.1.3     v dplyr   0.8.3
## v tidyr   1.0.0     v stringr 1.4.0
## v readr   1.3.1     v forcats 0.4.0
\end{verbatim}

\begin{verbatim}
## -- Conflicts ------------------------------------------------------------ tidyverse_conflicts() --
## x dplyr::filter() masks stats::filter()
## x dplyr::lag()    masks stats::lag()
\end{verbatim}

\begin{Shaded}
\begin{Highlighting}[]
\KeywordTok{library}\NormalTok{(cowplot)}
\end{Highlighting}
\end{Shaded}

\begin{verbatim}
## 
## ********************************************************
\end{verbatim}

\begin{verbatim}
## Note: As of version 1.0.0, cowplot does not change the
\end{verbatim}

\begin{verbatim}
##   default ggplot2 theme anymore. To recover the previous
\end{verbatim}

\begin{verbatim}
##   behavior, execute:
##   theme_set(theme_cowplot())
\end{verbatim}

\begin{verbatim}
## ********************************************************
\end{verbatim}

\begin{Shaded}
\begin{Highlighting}[]
\KeywordTok{library}\NormalTok{(viridis)}
\end{Highlighting}
\end{Shaded}

\begin{verbatim}
## Loading required package: viridisLite
\end{verbatim}

\begin{Shaded}
\begin{Highlighting}[]
\CommentTok{# Load data}
\NormalTok{NTL.PP.nut.tidy <-}\StringTok{ }\KeywordTok{read.csv}\NormalTok{(}\StringTok{"./Data/Processed/NTL-LTER_Lake_Chemistry_Nutrients_PeterPaul_Processed.csv"}\NormalTok{)}
\NormalTok{NTL.PP.nut.gathered <-}\StringTok{ }\KeywordTok{read.csv}\NormalTok{(}\StringTok{"./Data/Processed/NTL-LTER_Lake_Nutrients_PeterPaulGathered_Processed.csv"}\NormalTok{)}
\NormalTok{NIWO.Litter <-}\StringTok{ }\KeywordTok{read.csv}\NormalTok{(}\StringTok{"./Data/Processed/NEON_NIWO_Litter_mass_trap_Processed.csv"}\NormalTok{)}

\CommentTok{#2}
\KeywordTok{class}\NormalTok{(NTL.PP.nut.tidy}\OperatorTok{$}\NormalTok{sampledate) }\CommentTok{# Check date format}
\end{Highlighting}
\end{Shaded}

\begin{verbatim}
## [1] "factor"
\end{verbatim}

\begin{Shaded}
\begin{Highlighting}[]
\CommentTok{#Change dates to read as dates}
\NormalTok{NTL.PP.nut.tidy}\OperatorTok{$}\NormalTok{sampledate <-}\StringTok{ }
\StringTok{  }\KeywordTok{as.Date}\NormalTok{(NTL.PP.nut.tidy}\OperatorTok{$}\NormalTok{sampledate, }\DataTypeTok{format =} \StringTok{"%Y-%m-%d"}\NormalTok{)}
\NormalTok{NTL.PP.nut.gathered}\OperatorTok{$}\NormalTok{sampledate <-}\StringTok{ }
\StringTok{  }\KeywordTok{as.Date}\NormalTok{(NTL.PP.nut.gathered}\OperatorTok{$}\NormalTok{sampledate, }\DataTypeTok{format =} \StringTok{"%Y-%m-%d"}\NormalTok{)}
\NormalTok{NIWO.Litter}\OperatorTok{$}\NormalTok{collectDate <-}\StringTok{ }\KeywordTok{as.Date}\NormalTok{(NIWO.Litter}\OperatorTok{$}\NormalTok{collectDate, }\DataTypeTok{format =} \StringTok{"%Y-%m-%d"}\NormalTok{)}

\KeywordTok{class}\NormalTok{(NTL.PP.nut.tidy}\OperatorTok{$}\NormalTok{sampledate) }\CommentTok{# Verify date format}
\end{Highlighting}
\end{Shaded}

\begin{verbatim}
## [1] "Date"
\end{verbatim}

\hypertarget{define-your-theme}{%
\subsection{Define your theme}\label{define-your-theme}}

\begin{enumerate}
\def\labelenumi{\arabic{enumi}.}
\setcounter{enumi}{2}
\tightlist
\item
  Build a theme and set it as your default theme.
\end{enumerate}

\begin{Shaded}
\begin{Highlighting}[]
\NormalTok{myTheme <-}\StringTok{ }\KeywordTok{theme_classic}\NormalTok{(}\DataTypeTok{base_size =} \DecValTok{10}\NormalTok{) }\OperatorTok{+}
\StringTok{  }\KeywordTok{theme}\NormalTok{(}\DataTypeTok{axis.text =} \KeywordTok{element_text}\NormalTok{(}\DataTypeTok{color =} \StringTok{"black"}\NormalTok{), }
        \DataTypeTok{legend.position =} \StringTok{"top"}\NormalTok{) }\CommentTok{# Define a theme based off of the classic theme}

\KeywordTok{theme_set}\NormalTok{(myTheme) }\CommentTok{# Set defined theme to default}
\end{Highlighting}
\end{Shaded}

\hypertarget{create-graphs}{%
\subsection{Create graphs}\label{create-graphs}}

For numbers 4-7, create ggplot graphs and adjust aesthetics to follow
best practices for data visualization. Ensure your theme, color
palettes, axes, and additional aesthetics are edited accordingly.

\begin{enumerate}
\def\labelenumi{\arabic{enumi}.}
\setcounter{enumi}{3}
\tightlist
\item
  {[}NTL-LTER{]} Plot total phosphorus by phosphate, with separate
  aesthetics for Peter and Paul lakes. Add a line of best fit and color
  it black. Adjust your axes to hide extreme values.
\end{enumerate}

\begin{Shaded}
\begin{Highlighting}[]
\NormalTok{ppPlot1 <-}\StringTok{ }\KeywordTok{ggplot}\NormalTok{(NTL.PP.nut.tidy, }\KeywordTok{aes}\NormalTok{(}\DataTypeTok{x =}\NormalTok{ po4, }\DataTypeTok{y =}\NormalTok{ tp_ug)) }\OperatorTok{+}\StringTok{ }\CommentTok{# Plot of tp vs po4}
\StringTok{  }\KeywordTok{geom_point}\NormalTok{(}\KeywordTok{aes}\NormalTok{(}\DataTypeTok{shape =}\NormalTok{ lakename)) }\OperatorTok{+}\StringTok{ }\CommentTok{# Scatter plot with lakes as different shapes}
\StringTok{  }\KeywordTok{geom_smooth}\NormalTok{(}\DataTypeTok{method =}\NormalTok{ lm, }\DataTypeTok{color =} \StringTok{"black"}\NormalTok{) }\OperatorTok{+}\StringTok{ }\CommentTok{# Add linear best fit line}
\StringTok{  }\KeywordTok{xlim}\NormalTok{(}\DecValTok{0}\NormalTok{,}\DecValTok{50}\NormalTok{) }\OperatorTok{+}\StringTok{ }\CommentTok{# Limit x-axis to hide outliers}
\StringTok{  }\KeywordTok{labs}\NormalTok{(}\DataTypeTok{x =} \StringTok{"Phosphate"}\NormalTok{, }\DataTypeTok{y =} \KeywordTok{expression}\NormalTok{(}\KeywordTok{paste}\NormalTok{(}\StringTok{"Total Phosphorous ( "}\NormalTok{, mu, }\StringTok{"g/L)"}\NormalTok{)),}
       \DataTypeTok{shape =} \StringTok{""}\NormalTok{) }\CommentTok{# Change axis and legend labels}

\KeywordTok{print}\NormalTok{(ppPlot1 )}
\end{Highlighting}
\end{Shaded}

\begin{verbatim}
## Warning: Removed 21947 rows containing non-finite values (stat_smooth).
\end{verbatim}

\begin{verbatim}
## Warning: Removed 21947 rows containing missing values (geom_point).
\end{verbatim}

\includegraphics{Hancock_A05_DataVisualization_files/figure-latex/unnamed-chunk-3-1.pdf}

\begin{enumerate}
\def\labelenumi{\arabic{enumi}.}
\setcounter{enumi}{4}
\tightlist
\item
  {[}NTL-LTER{]} Make three separate boxplots of (a) temperature, (b)
  TP, and (c) TN, with month as the x axis and lake as a color
  aesthetic. Then, create a cowplot that combines the three graphs. Make
  sure that only one legend is present and that graph axes are aligned.
\end{enumerate}

\begin{Shaded}
\begin{Highlighting}[]
\CommentTok{# Make Temperature boxplots}
\NormalTok{ppTempPlot <-}\StringTok{ }\KeywordTok{ggplot}\NormalTok{(NTL.PP.nut.tidy) }\OperatorTok{+}
\StringTok{  }\KeywordTok{geom_boxplot}\NormalTok{(}\KeywordTok{aes}\NormalTok{(}\DataTypeTok{x =} \KeywordTok{as.factor}\NormalTok{(month), }\DataTypeTok{y =}\NormalTok{ temperature_C, }\DataTypeTok{color =}\NormalTok{ lakename)) }\OperatorTok{+}
\StringTok{  }\KeywordTok{labs}\NormalTok{(}\DataTypeTok{x =} \StringTok{"Month"}\NormalTok{, }\DataTypeTok{y =} \StringTok{"Temperature (C)"}\NormalTok{, }\DataTypeTok{color =} \StringTok{""}\NormalTok{) }\OperatorTok{+}
\StringTok{  }\KeywordTok{scale_color_manual}\NormalTok{(}\DataTypeTok{values =} \KeywordTok{c}\NormalTok{(}\StringTok{"#0c2c84"}\NormalTok{, }\StringTok{"#ea6827ff"}\NormalTok{)) }\CommentTok{# Set colors}

\CommentTok{#ppTempPlot # Used for debugging}

\CommentTok{#Make Phosphorous boxplots}
\NormalTok{ppTPPlot <-}\StringTok{ }\KeywordTok{ggplot}\NormalTok{(NTL.PP.nut.tidy) }\OperatorTok{+}
\StringTok{  }\KeywordTok{geom_boxplot}\NormalTok{(}\KeywordTok{aes}\NormalTok{(}\DataTypeTok{x =} \KeywordTok{as.factor}\NormalTok{(month), }\DataTypeTok{y =}\NormalTok{ tp_ug, }\DataTypeTok{color =}\NormalTok{ lakename)) }\OperatorTok{+}
\StringTok{  }\KeywordTok{labs}\NormalTok{(}\DataTypeTok{x =} \StringTok{"Month"}\NormalTok{, }\DataTypeTok{y =} \KeywordTok{expression}\NormalTok{(}\KeywordTok{paste}\NormalTok{(}\StringTok{"Total Phosphorous ( "}\NormalTok{, mu, }\StringTok{"g/L)"}\NormalTok{))) }\OperatorTok{+}
\StringTok{  }\KeywordTok{scale_color_manual}\NormalTok{(}\DataTypeTok{values =} \KeywordTok{c}\NormalTok{(}\StringTok{"#0c2c84"}\NormalTok{, }\StringTok{"#ea6827ff"}\NormalTok{)) }\CommentTok{# Set colors}

\CommentTok{# Make Nitrogen boxplots}
\NormalTok{ppTNPlot <-}\StringTok{ }\KeywordTok{ggplot}\NormalTok{(NTL.PP.nut.tidy) }\OperatorTok{+}
\StringTok{  }\KeywordTok{geom_boxplot}\NormalTok{(}\KeywordTok{aes}\NormalTok{(}\DataTypeTok{x =} \KeywordTok{as.factor}\NormalTok{(month), }\DataTypeTok{y =}\NormalTok{ tn_ug, }\DataTypeTok{color =}\NormalTok{ lakename)) }\OperatorTok{+}
\StringTok{  }\KeywordTok{labs}\NormalTok{(}\DataTypeTok{x =} \StringTok{"Month"}\NormalTok{, }\DataTypeTok{y =} \KeywordTok{expression}\NormalTok{(}\KeywordTok{paste}\NormalTok{(}\StringTok{"Total Nitrogen ( "}\NormalTok{, mu, }\StringTok{"g/L)"}\NormalTok{))) }\OperatorTok{+}
\StringTok{  }\KeywordTok{scale_color_manual}\NormalTok{(}\DataTypeTok{values =} \KeywordTok{c}\NormalTok{(}\StringTok{"#0c2c84"}\NormalTok{, }\StringTok{"#ea6827ff"}\NormalTok{)) }\CommentTok{# Set colors}

\CommentTok{# Extract legend from one of the plots to include in combined cowplot}
\NormalTok{ppLegend <-}\StringTok{ }\KeywordTok{get_legend}\NormalTok{(}
\NormalTok{    ppTempPlot }\OperatorTok{+}\StringTok{ }\KeywordTok{theme}\NormalTok{(}\DataTypeTok{legend.box.margin =} \KeywordTok{margin}\NormalTok{(}\DecValTok{0}\NormalTok{, }\DecValTok{0}\NormalTok{, }\DecValTok{0}\NormalTok{, }\DecValTok{12}\NormalTok{)))}
\end{Highlighting}
\end{Shaded}

\begin{verbatim}
## Warning: Removed 3566 rows containing non-finite values (stat_boxplot).
\end{verbatim}

\begin{Shaded}
\begin{Highlighting}[]
\CommentTok{# Create a plot of the three sets of boxplots}
\NormalTok{ppPlotGrid <-}\StringTok{ }\KeywordTok{plot_grid}\NormalTok{(ppTempPlot }\OperatorTok{+}\StringTok{ }\KeywordTok{theme}\NormalTok{(}\DataTypeTok{legend.position=}\StringTok{"none"}\NormalTok{), }\CommentTok{# Remove legends}
\NormalTok{                        ppTPPlot }\OperatorTok{+}\StringTok{ }\KeywordTok{theme}\NormalTok{(}\DataTypeTok{legend.position=}\StringTok{"none"}\NormalTok{),}
\NormalTok{                        ppTNPlot }\OperatorTok{+}\StringTok{ }\KeywordTok{theme}\NormalTok{(}\DataTypeTok{legend.position=}\StringTok{"none"}\NormalTok{),}
                        \DataTypeTok{align =} \StringTok{'v'}\NormalTok{, }\DataTypeTok{ncol =} \DecValTok{1}\NormalTok{) }\CommentTok{# Align axes, limit to one column}
\end{Highlighting}
\end{Shaded}

\begin{verbatim}
## Warning: Removed 3566 rows containing non-finite values (stat_boxplot).
\end{verbatim}

\begin{verbatim}
## Warning: Removed 20729 rows containing non-finite values (stat_boxplot).
\end{verbatim}

\begin{verbatim}
## Warning: Removed 21583 rows containing non-finite values (stat_boxplot).
\end{verbatim}

\begin{Shaded}
\begin{Highlighting}[]
\CommentTok{# Create plot with boxplots and legend}
\NormalTok{ppPlot2 <-}\StringTok{ }\KeywordTok{plot_grid}\NormalTok{(ppPlotGrid, ppLegend, }\DataTypeTok{ncol =} \DecValTok{1}\NormalTok{, }\DataTypeTok{rel_heights =} \KeywordTok{c}\NormalTok{(}\DecValTok{3}\NormalTok{,.}\DecValTok{3}\NormalTok{))}
\KeywordTok{print}\NormalTok{(ppPlot2)}
\end{Highlighting}
\end{Shaded}

\includegraphics{Hancock_A05_DataVisualization_files/figure-latex/unnamed-chunk-4-1.pdf}

Question: What do you observe about the variables of interest over
seasons and between lakes?

\begin{quote}
Answer: As would be expected, the median temperature for both lakes
rises during the summer months and is lower towards both ends of the
year (in the winter). There does not appear to be an appreciable
difference between the temperatures of the two lakes. Total phosphorous
is similar between the two lakes, but it seems like Peter Lake tends to
have a slightly higher concentration, especially in later summer months.
A similar pattern is seen regarding nitrogen concentration. Phosphorous
and Nitrogen measurements were not taken for colder months for either
lake.
\end{quote}

\begin{enumerate}
\def\labelenumi{\arabic{enumi}.}
\setcounter{enumi}{5}
\item
  {[}Niwot Ridge{]} Plot a subset of the litter dataset by displaying
  only the ``Needles'' functional group. Plot the dry mass of needle
  litter by date and separate by NLCD class with a color aesthetic. (no
  need to adjust the name of each land use)
\item
  {[}Niwot Ridge{]} Now, plot the same plot but with NLCD classes
  separated into three facets rather than separated by color.
\end{enumerate}

\begin{Shaded}
\begin{Highlighting}[]
\CommentTok{#6 NCLD classes separated by color}
\NormalTok{nrPlot1 <-}\StringTok{ }\KeywordTok{ggplot}\NormalTok{(}\KeywordTok{subset}\NormalTok{(NIWO.Litter, functionalGroup }\OperatorTok{==}\StringTok{ "Needles"}\NormalTok{)) }\OperatorTok{+}
\StringTok{  }\KeywordTok{geom_point}\NormalTok{(}\KeywordTok{aes}\NormalTok{(}\DataTypeTok{x =}\NormalTok{ collectDate, }\DataTypeTok{y =}\NormalTok{ dryMass, }\DataTypeTok{color =}\NormalTok{ nlcdClass)) }\OperatorTok{+}
\StringTok{  }\KeywordTok{labs}\NormalTok{(}\DataTypeTok{x =} \StringTok{"Date Collected"}\NormalTok{, }\DataTypeTok{y =} \StringTok{"Dry Mass (g)"}\NormalTok{, }\DataTypeTok{color =} \StringTok{""}\NormalTok{) }\OperatorTok{+}
\StringTok{  }\KeywordTok{scale_color_viridis}\NormalTok{(}\DataTypeTok{discrete =} \OtherTok{TRUE}\NormalTok{, }\DataTypeTok{end =} \FloatTok{0.8}\NormalTok{) }\OperatorTok{+}\StringTok{ }\CommentTok{# Set color scheme}
\StringTok{  }\KeywordTok{scale_x_date}\NormalTok{(}\DataTypeTok{date_breaks =} \StringTok{"6 months"}\NormalTok{) }\CommentTok{# Change the breaks on the x-axis for readability}

\KeywordTok{print}\NormalTok{(nrPlot1)}
\end{Highlighting}
\end{Shaded}

\includegraphics{Hancock_A05_DataVisualization_files/figure-latex/unnamed-chunk-5-1.pdf}

\begin{Shaded}
\begin{Highlighting}[]
\CommentTok{#nrPlot2 <- ggplot(subset(NIWO.Litter, functionalGroup == "Needles")) +}
\CommentTok{#  geom_boxplot(aes(x = as.factor(collectDate), y = dryMass, color = nlcdClass)) +}
\CommentTok{#  scale_color_viridis(discrete = TRUE, end = 0.8)}

\CommentTok{#print(nrPlot2)}

\CommentTok{# 7 NLCD Classes separated on different plots}
\NormalTok{nrPlot3 <-}\StringTok{ }\KeywordTok{ggplot}\NormalTok{(}\KeywordTok{subset}\NormalTok{(NIWO.Litter, functionalGroup }\OperatorTok{==}\StringTok{ "Needles"}\NormalTok{)) }\OperatorTok{+}
\StringTok{  }\KeywordTok{geom_point}\NormalTok{(}\KeywordTok{aes}\NormalTok{(}\DataTypeTok{x =}\NormalTok{ collectDate, }\DataTypeTok{y =}\NormalTok{ dryMass)) }\OperatorTok{+}
\StringTok{  }\KeywordTok{facet_grid}\NormalTok{(}\KeywordTok{vars}\NormalTok{(nlcdClass)) }\OperatorTok{+}\StringTok{ }\CommentTok{# Create facet grid based on NCLD Class}
\StringTok{  }\KeywordTok{scale_x_date}\NormalTok{(}\DataTypeTok{date_breaks =} \StringTok{"6 months"}\NormalTok{) }\OperatorTok{+}\StringTok{ }\CommentTok{# Change the breaks on the x-axis for readability}
\StringTok{  }\KeywordTok{labs}\NormalTok{(}\DataTypeTok{x =} \StringTok{"Date Collected"}\NormalTok{, }\DataTypeTok{y =} \StringTok{"Dry Mass (g)"}\NormalTok{)}

\KeywordTok{print}\NormalTok{(nrPlot3)}
\end{Highlighting}
\end{Shaded}

\includegraphics{Hancock_A05_DataVisualization_files/figure-latex/unnamed-chunk-5-2.pdf}

Question: Which of these plots (6 vs.~7) do you think is more effective,
and why?

\begin{quote}
Answer: Plot 6 seems more effective because you can see the points
side-by-side and on top of each other on the same set of axes. This
format allows you to easily see how they compare to each other. In plot
7, it is hard to really determine if the points in one NLCD class are
relatively higher or lower than a different class unless there is
extreme variation.
\end{quote}

\end{document}
